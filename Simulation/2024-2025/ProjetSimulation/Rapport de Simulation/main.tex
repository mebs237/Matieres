\documentclass[a4paper,12pt]{beamer}

\usetheme{Boadilla}

% Slide automatique à chaque début de section
\AtBeginSection[]{
  \begin{frame}
    \centering
    {\color{blue} \LARGE \bfseries \insertsection}
  \end{frame}
}

% Titre de sous-section affiché dans la marge (pas de slide)
\AtBeginSubsection[]{
}

%importation des csv et leur lecture
\usepackage{csvsimple}
\usepackage{booktabs}

\setbeamertemplate{frametitle continuation}{}

\usepackage{xcolor}
\usepackage{tikz}
\usetikzlibrary{calc}
\usepackage{microtype}
\usepackage[utf8]{inputenc}
\usepackage[OT1]{fontenc}
\usepackage{siunitx}
\usepackage[french]{babel}
\usepackage[T1]{fontenc}
\usepackage{amsmath , amsthm , amscd , amsfonts , amssymb}
\usepackage{geometry}

% Commandes pour les algorithmes
\renewcommand{\listalgorithmname}{Liste des algorithmes}
\floatname{algorithm}{Algorithme}
\renewcommand{\algorithmicreturn}{\textbf{retourne}}
\renewcommand{\algorithmicprocedure}{\textbf{procédure}}
% \renewcommand{\Not}{\textbf{non}\ }
\renewcommand{\And}{\textbf{et}\ }
% \renewcommand{\Or}{\textbf{ou}\ }
\renewcommand{\algorithmicrequire}{\textbf{Entrées:}}
\renewcommand{\algorithmicensure}{\textbf{Sorties:}}
%\renewcommand{\algorithmiccomment}[1]{\{#1\}}
\renewcommand{\algorithmicend}{\textbf{fin}}
\renewcommand{\algorithmicif}{\textbf{si}}
\renewcommand{\algorithmicthen}{\textbf{alors}}
\renewcommand{\algorithmicelse}{\textbf{sinon}}
\renewcommand{\algorithmicfor}{\textbf{pour}}
\renewcommand{\algorithmicforall}{\textbf{pour tout}}
\renewcommand{\algorithmicdo}{\textbf{faire}}
\renewcommand{\algorithmicwhile}{\textbf{tant que}}
\newcommand{\algorithmicelsif}{\algorithmicelse\ \algorithmicif}
\newcommand{\algorithmicendif}{\algorithmicend\ \algorithmicif}
\newcommand{\algorithmicendfor}{\algorithmicend\ \algorithmicfor}

\algnewcommand\algorithmicforeach{\textbf{pour chaque}}
\algdef{S}[FOR]{ForEach}[1]{\algorithmicforeach\ #1\ \algorithmicdo}     

\theoremstyle{definition}
\newtheorem{definition}{Définition}[section]

\newtheorem*{example}{Exemple}
\newtheorem*{remark}{Remarque}

\setlength{\parskip}{4mm}

\begin{document}

\begin{titlepage}
    \begin{center}
        
        {\Large Université de Mons}\\[1ex]
        {\Large Faculté Des Sciences}\\[1ex]
        {\Large Département d'Informatique}\\[1ex]
        
        \newcommand{\HRule}{\rule{\linewidth}{0.3mm}}
        % Title
        \HRule \\[0.3cm]
        { \LARGE \bfseries Rapport du projet de Simulation\\[0.3cm]}
        { \LARGE \bfseries Etude et Générateur pseudo-aléatoire avec les décimales de $e$\\[0.3cm]}
        \HRule \\[1.5cm]
        
        % Author and supervisor
        \begin{minipage}[t]{0.45\textwidth}
            \begin{flushleft} \large
                \emph{Professeur:}\\
                Alain \textsc{Buys}

            \end{flushleft}
        \end{minipage}
        \begin{minipage}[t]{0.45\textwidth}
            \begin{flushright} \large
                \emph{Auteurs:} \\
                \textsc{Prestonne MEBOU} \\
                \textsc{Wilfried NJANJA} \\
                

                \vspace{2mm}

            \end{flushright}
        \end{minipage}\\[2ex]
        
        \vfill
        
        % Bottom of the page
        \begin{center}
            \begin{tabular}[t]{c c c}
                \includegraphics[height=1.5cm]{images/UMONS-Logo.jpg} &
                \includegraphics[height=1.5cm]{images/FS-Logo.jpg} &
                %\hspace{0.3cm} &
            \end{tabular}
        \end{center}~\\
        
        {\large Année académique 2024-2025}
        
    \end{center}
\end{titlepage}



\tableofcontents
\newpage

\section{Introduction}
\addcontentsline{toc}{section}{Introduction}
La génération de nombres pseudo-aléatoires est un enjeu crucial dans de nombreux domaines tels que la cryptographie, les systèmes bancaires ou la cybersécurité. Ces applications exigent des séquences numériques qui \emph{imitent} parfaitement l'aléatoire. Pour garantir cette qualité, des tests statistiques rigoureux sont indispensables.

Dans le cadre de ce projet, nous avons étudié les décimales du nombre $e$ (2\,000\,000 de décimales fournies) pour évaluer leur caractère pseudo-aléatoire. Notre démarche s'est articulée en deux temps~:
\begin{itemize}[leftmargin=*]
    \item Analyse directe des décimales via des tests statistiques ($\chi^2$, Kolmogorov-Smirnov).
    \item Construction d'un générateur de loi uniforme $[0,1[$ basé sur ces décimales, comparé ensuite au générateur natif de Python.
\end{itemize}

Pour implémenter ces tests, nous avons adopté une approche orientée objet, permettant une réutilisation aisée du code.Bien que nous ayons implémenté plusieurs tests(que vous trouverez dans le code fourni), nous avons opté pa ce rapport de vous présenter notamment les résultats des tests du $\chi^2$, du collectionneur de coupons et de Kolmogorov-Smirnov.

\clearpage

\input{rapport/section/étude_decimal_e}
\input{rapport/section/etude_générateur}

\input{rapport/section/comparaisons_génarteur}
\clearpage
\section*{Conclusion}

\addcontentsline{toc}{section}{Conclusion}

En conclusion, notre travail sur la résolution du problème de MapOverlay a été assez satisfaisant, car nous avons réussi à développer un algorithme de résolution qui gère de manière relativement efficace le problème de superposition des cartes. Notre approche a permis de traiter efficacement la plupart des cas, offrant ainsi une solution viable pour de nombreuses applications.

Cependant, malgré les progrès réalisés, notre algorithme présente encore des limitations. Notamment, la gestion de certains cas que, par défaut de temps, nous n'avons pas clairement pu identifier, reste perfectible. De plus, des améliorations pourraient être apportées pour permettre le calcul des intersections entre plus de deux cartes, ainsi que pour tracer une carte résultante de la construction des différents points. De nombreuses optimisations pourraient également être apportées à notre interface graphique dans le but de la rendre beaucoup plus optimale dans son fonctionnement entre ses différentes fonctionnalités.

Néanmoins, nous sommes convaincus que notre algorithme constitue une contribution significative à la résolution de ce problème complexe. Il fournit une base solide pour des recherches futures visant à affiner et à améliorer la précision et la robustesse de la solution. De plus, il ouvre la voie à de nouvelles applications dans des domaines tels que la cartographie, la navigation et la visualisation de données géospatiales.








\section{Conclusion}
Le modèle final (\textbf{XGBoost}) atteint une précision de \textbf{4.21 RMSE}, avec comme drivers principaux les indicateurs fiscaux et démographiques. Les résultats suggèrent que \textbf{[\%TODO : insérer une interprétation politique]}. Pour aller plus loin :
\begin{itemize}
    \item Intégrer des données de campagne électorale
    \item Affiner la granularité spatiale (IRIS au lieu de communes)
\end{itemize}


% printing bibliography items
%\addcontentsline{toc}{section}{Source}
%\printbibheading[title={Source}]
%\printbibliography[title={Source}]
\nocite{*}
\bibliographystyle{plain} % Choose your style
\bibliography{book}

\end{document}
