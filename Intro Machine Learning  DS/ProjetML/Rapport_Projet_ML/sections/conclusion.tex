\clearpage
\section*{Conclusion}

\addcontentsline{toc}{section}{Conclusion}

En conclusion, notre travail sur la résolution du problème de MapOverlay a été assez satisfaisant, car nous avons réussi à développer un algorithme de résolution qui gère de manière relativement efficace le problème de superposition des cartes. Notre approche a permis de traiter efficacement la plupart des cas, offrant ainsi une solution viable pour de nombreuses applications.

Cependant, malgré les progrès réalisés, notre algorithme présente encore des limitations. Notamment, la gestion de certains cas que, par défaut de temps, nous n'avons pas clairement pu identifier, reste perfectible. De plus, des améliorations pourraient être apportées pour permettre le calcul des intersections entre plus de deux cartes, ainsi que pour tracer une carte résultante de la construction des différents points. De nombreuses optimisations pourraient également être apportées à notre interface graphique dans le but de la rendre beaucoup plus optimale dans son fonctionnement entre ses différentes fonctionnalités.

Néanmoins, nous sommes convaincus que notre algorithme constitue une contribution significative à la résolution de ce problème complexe. Il fournit une base solide pour des recherches futures visant à affiner et à améliorer la précision et la robustesse de la solution. De plus, il ouvre la voie à de nouvelles applications dans des domaines tels que la cartographie, la navigation et la visualisation de données géospatiales.








\section{Conclusion}
Le modèle final (\textbf{XGBoost}) atteint une précision de \textbf{4.21 RMSE}, avec comme drivers principaux les indicateurs fiscaux et démographiques. Les résultats suggèrent que \textbf{[\%TODO : insérer une interprétation politique]}. Pour aller plus loin :
\begin{itemize}
    \item Intégrer des données de campagne électorale
    \item Affiner la granularité spatiale (IRIS au lieu de communes)
\end{itemize}
